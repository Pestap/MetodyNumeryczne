\documentclass{article}
\usepackage[utf8]{inputenc}
\usepackage{polski}
\usepackage{graphicx}
\usepackage[margin= 2.5cm]{geometry}
\usepackage{amsmath}
\usepackage{float}
\usepackage[section]{placeins}

\title{Wskaźnik MACD}
\author{Piotr Pesta, 184531}
\date{Marzec 2022}
\setlength{\parindent}{20pt}
\graphicspath{{./Graphs/}}

\begin{document}
\maketitle

\section{Wstęp}

    Celem projektu było zaimplementowanie wskaźnika MACD (Moving Average Convergence Divergence), który pozwala na podstawie analizy danych historycznych
    wyznaczać momenty kupna i sprzedaży udziałów. Wskaźnik ten składa się z dwóch krzywych: MACD oraz Signal.
    Implementacji dokonałem w języku Python z wykorzystaniem bibliotek pandas oraz matplotlib.
    Dane wejściowe to plik .csv o zawierający informacje o cenach udziałów z konkretnych dni. 
    Jako wartość udziałów w danym dniu przyjąłem cenę zamknięcia z tego dnia. Do algorytmu obracającego akcjami podajemy dane z ok. 1000 dni.

\section{Analiza zadania}

    W celu określenia na podstawie wskaźnika MACD momentów kupna i sprzedaży akcji należy obliczyć
    dwie średnie kroczące z wartości udziałów:
    \begin{itemize}
        \item krótkookresową - 12 okresów
        \item długookresową - 26 okresów
    \end{itemize}

    \noindent Następnie wartość MACD uzyskujemy odejmując średnią długookresową od krótkookresowej.
    W celu obliczenia Signal należy obliczyć średnią krocząca z MACD z 9 okresów. 


    Do obliczenia średnich kroczących skorzystałem z poniższego wzoru:
 
        

    \begin{align*} 
        EMA_{N} &= \frac{p_{0} + (1-\alpha)p_{1} + (1-\alpha)^2p_{2} + \ldots + (1-\alpha)^Np_{N}}{1 + (1-\alpha)+(1-\alpha)^2 + \ldots + (1-\alpha)^N} \\
        \mathit{Gdzie:} & \\
        \alpha &= \frac{2}{N + 1} \\
        N &- liczba \; okres \mathit{ó} w \\
        p_{i} &- pr\mathit{ó}bka \; z \; i-tego \;dnia, \;p_{0}\; - \; pr\mathit{ó}bka \;z \;aktualnego \;dnia, \;p_{N} \;-\;pr\mathit{ó}bka \;sprzed \;N \;dni.
    \end{align*}

    Punkty, w których krzywa MACD przecina Signal od dołu są sygnałem do kupna akcji oraz zapowiadają trend wzrostowy.
    Natomiast punkty, w których MACD przecina Signal od góry, są sygnałem sprzedaży akcji i zapowiedzią trendu spadkowego. 
    
    Kod zadania podzieliłem na dwa pliki. Jeden z nich zawiera wszystkie funckje, z których korzystałem do obliczeń, inwestycji oraz rysowania wykresu,
     a drugi obrazuje kolejność działań poprzez wywoływanie w odpowiedniej kolejności wcześniej wspomnianych funkcji.
\section{Analiza skuteczności i przykłady działania}
    \begin{figure}[H]
        \noindent\makebox[\textwidth]{\includegraphics[width=\paperwidth]{CD Projekt - 4 lata.png}}
        \caption{Wykres ceny udziałów CDR oraz obliczonych na ich podstawie MACD oraz Signal. Wykres ten składa się z ok. 900 cen zamknięcia.
         W celu zachowania czytelności wykresu nie naniosłem na niego linii wyznaczających momenty kupna i sprzedaży.}
    \end{figure}

    Analizując powyższy wykres możemy dojść do kilku wniosków:
    \begin{itemize}
        \item dla powolnych zmian wskaźnik MACD jest skuteczny i pozwala osiągnąć zyski. Jest to dobrze widoczne na wykresie w okresie od 12.03.2022 do 15.05.2020 oraz na lewej części wykresu.
        \item MACD reaguje z opóźnieniem na nagłe zmiany cen udziałów. Możemy to zauważyć na wykresie w okolicach 21.01.2021, kiedy wystąpił pik wartości udziałów.
         Na pierwszy rzut oka może wydawać się, że wskaźnik zareagował prawidłowo, jednak gdy przyjrzymy się dokładniej to zauważymy, że sygnał sprzedaży jest spóźniony i znajduję się poza pikiem.
         Wskaźnik nie nadążył za zmianą i okazja do zarobku przepadła.
    \end{itemize}
\newpage
\section{Analiza przydatności MACD do inwestowania}

    Analziując zachowanie algorytmu dla różnych wejść, możemy zauważyć, że najprostszy algorytm,
    kupujący i sprzedający udziały w momentach wyznaczonych jedynie przez MACD, nie osiąga zadowalających rezultatów. Dla niekótrych indeksów osiąga
    bardzo dobre wyniki, ale w większości przypadków wartość naszego portfela na końcu testu jest niższa niż na początku.

    
    Efektywność algorytmu określamy jako stosunek wartości naszego portfela
    do jego stanu po działaniu algorytmu na podstawie poniższego wzoru:

    \begin{align*} 
        zysk = \frac{\mathit{ilośćAkcji}(N) \cdot \mathit{cenaAkcji}(N) + \mathit{gotówka}(N)}{1000 \cdot \mathit{cenaAkcji}(0)}
    \end{align*}


    \noindent Wartość poniżej 1 będzie oznaczać stratę, a powyżej zysk.\\
    \noindent Przykładowe wyniki:

    \subsection{Algorytm podstawowy}
    \begin{figure}[H]
        \noindent\makebox[\textwidth]{\includegraphics{AnalizaCDPDługi.png}}
        \caption{CDP w latach 2012-2022 - ponad 9-krotny zysk. Jest to przypadek potwierdzający
        prydatność wskaźnika MACD w przypadku inwestycji długoterminowych dla stabilnych akcji.}
    \end{figure}
    \begin{figure}[H]
        \noindent\makebox[\textwidth]{\includegraphics{AnalizaOrlen.png}}
        \caption{PKN Orlen 2018-2022 - zysk ok. 1,2 krotny. Kolejne potwierdzenie dobrego działania algorytmu dla stablinych akcji i długoterminowych inwestycji}
    \end{figure}
    \begin{figure}[H]
        \noindent\makebox[\textwidth]{\includegraphics{AnalizaWIG20.png}}
        \caption{WIG20 2015-2022 - w tym przypadku mamy do czynienia ze stratę, co pokazuje, że wskaźnik MACD nie zawsze jest skuteczny}
    \end{figure}
    \begin{figure}[H]
        \noindent\makebox[\textwidth]{\includegraphics{AnalizaCDP.png}}
        \caption{CDP w latach 2018-2022 - ok. 1.5 krotny zysk}
    \end{figure}
    

    \subsection{Algorytm rozszerzony}
    Główną wadą prostego algorytmu jest jego wolna reakcja na nagłe zmiany wartości akcji. Takie zmiany są często niemożliwe do przewidzenia, gdyż mogą wynikać 
    ze zdarzeń, na które nie ma wpływu rynek, takich jak np. wojna, katastrofa klimatyczna. Skupiłem się więc na ulepszeniu działania algorytmu dla długoterminowych inwestycji.
    Dodałem do niego warunek, który sprzedaje akcje tylko wtedy kiedy kupił je w niższej cenia oraz sprawdzający czy w ostatnim miesiącu dokonał sprzedaży.
    Jeżeli w ciągu tego miesiąca algorytm nie sprzedał akcji, to oznacza to, że cena akcji spadła na dłużej i należy wznowić inwestowanie - zresetować cenę ostatniego zakupu.
    \begin{figure}[H]
        \noindent\makebox[\textwidth]{\includegraphics{WynikAsseco.png}}
        \caption{Asseco Poland 2018-2022 - widać przewagę algorytmu rozszerzonego,
        jego zastosowanie pozwoliło osiągnąć zysk}
    \end{figure}
    \begin{figure}[H]
        \noindent\makebox[\textwidth]{\includegraphics{WynikWig.png}}
        \caption{WIG20 2015-2022 - drobna przewaga algorytmu rozszerzonego, ale pomimo tego strata}
    \end{figure}
    \begin{figure}[H]
        \noindent\makebox[\textwidth]{\includegraphics{WynikCDP.png}}
        \caption{CD Projekt 2015-2022 - zwiększenie zysków}
    \end{figure}

    \begin{figure}[H]
        \noindent\makebox[\textwidth]{\includegraphics{WynikOrlen.png}}
        \caption{Orlen 2018 - 2022 - działanie odwrotne do zamierzonego - algorytm 'ulepszony' okazał się gorszy}
    \end{figure}
    \begin{figure}[H]
        \noindent\makebox[\textwidth]{\includegraphics{WynikCDP2.png}}
        \caption{CDP 2018 - 2022 - również rezultat niepożądany}
    \end{figure}
    \begin{figure}[H]
        \noindent\makebox[\textwidth]{\includegraphics{WynikPKO.png}}
        \caption{PKO BP 2016 - 2022 - rezultat gorszy od algorytmu podstawowego}
    \end{figure}

    Skuteczność modyfikacji okazała się być słaba a rezultaty w wielu przypadkach były gorsze.
    Gdy przyjrzymy się jednak ogólnym wynikom to są one zadowalające, najgorszy rezultat to około 80\%,
    a więc algorytm przeciwdziała dużym stratom i tak jak w przypadku CD Projekt pozwala osiągnąć duże zyski, 
    jeżeli dobrze wybierzemy inwestycje.

\section{Podsumowanie}
    Wskaźnik MACD nie nadaje się do inwestycji krótkoterminowych z powodu opóźnionej rekacji na
    zmianę cen akcji, ale jest dobrym narzędziem do inwestycji długoterminowych, szczególnie w akcje stabilne,
    które rzadko ulegają dużym wahaniom. 


\end{document}